%% @Introduce  : 
%% @File       : TimeLine.tex
%% @Author     : ryrl
%% @Email      : ryrl970311@gmail.com
%% @Time       : 2025/01/16 15:16
%% @Description:

\begin{enumerate}[label=(\arabic*)]
    \setlength{\itemsep}{0pt}
    \item \textbf{Months 1-3}: Literature review on GNNs, causal inference, and gene regulatory networks. Collection of available single-cell omics datasets for benchmarking and validation \footnote{Literature review will be an ongoing process throughout the project.}.
    \item  \textbf{Months 4-7}: Implement the model architecture, including selection of hyperparameters (layers, attention heads, activation functions).
    \item \textbf{Months 8-11}: Model training and preliminary evaluation on a subset of the data. Explore different causal inference techniques (additive noise model, do-calculus).
    \item \textbf{Months 12-14}: Refine the GNN model and causal inference methods based on initial results. Optimize model performance and address challenges related to cyclic relationships in GRNs.
    \item  \textbf{Months 15-17}: Conduct comprehensive model evaluation using various metrics (accuracy, precision, recall, F1-score, AUROC, AUPRC). Compare the performance of the proposed methodology with existing GRN inference methods. 
    \item \textbf{Months 18-20}: Focus on model interpretation and visualization. Develop methods to interpret the learned gene embeddings and identify important features contributing to causal relationships.
    \item \textbf{Months 21-24}: Validate the inferred GRN using independent datasets or experimental validation techniques.
    \item \textbf{Months 25-27}: Write the thesis, incorporating all research findings, methodology, and analysis.
\end{enumerate}